% Tout ce qui est compris entre le caractère % et une fin de ligne est un
% commentaire ignoré par LaTeX

% Un fichier LaTeX commence par une commande \documentclass qui déclare le type
% de document

\documentclass{article}

% Ici, et jusqu'au \begin{document} ci-dessous, c'est le préambule : on y met
% les déclarations de style, les définitions de macros, etc.

%%%% Personnalisation du style défaut

% Fichiers de style utilisés ; les déclarations se font au moyen de la commande
% \usepackage

\usepackage{fullpage} % Agrandit les dimensions du texte (hauteur, largeur,
                      % etc.) par rapport à celles par défaut. Attention
                      % ce package ne se trouve pas dans toutes les
                      % distributions LaTeX

\usepackage[french]{babel} % Pour adopter les règles de typographie française

\usepackage[T1]{fontenc} % Pour que LaTeX comprenne les caractères accentués ;
                         % norme iso-8859, cela risque de ne pas marcher avec
                         % des fichiers créés sous windows

\usepackage{amsmath} % Les bibliothèques LaTeX de l'American
                     % Mathematical Society sont pleines de macros
                     % intéressantes (voire indispensables).

%%%% Déclaration d'environnements

% Les définitions, théorèmes et autres lemmes, corollaires, propositions,
% exemples, se déclarent au moyen de \newtheorem

\newtheorem{de}{Définition}[subsection] % les définitions et les théorèmes sont
\newtheorem{theo}{Théorème}[section]    % numérotés par section
\newtheorem{prop}[theo]{Proposition}    % Les propositions ont le même compteur
                                        % que les théorèmes

%%%% Notations

% Ici l'on définit les notations que l'on utilisera, au moyen de la macro
% \newcommand

\newcommand{\ordmult}{\mathop{.}} % \ordmult est l'opérateur de multiplication
                                  % d'ordinaux

% Voici une notation qui prend des paramètres

\newcommand{\expo}[2]{#1^{#2}}

%%%% déclarations pour le titre

\author{Laurent Regnier\\
        Université de la Méditerranée
        \and 
        Roland Reinger\\
        Institut Énarretidem
}

\title{Suites de Goodstein et hydre de Lerne\\
       \small TER de maîtrise}
%\date{}

%%%% fin du préambule, on passe au contenu : tout le texte entre
%%%% \begin{document} et \end{document} 

\begin{document}
\maketitle

% Un petit résumé
\begin{abstract}
  Dans ce rapport on va montrer qu'aussi surprenant que cela puisse paraître,
  toute suite de Goodstein est ultimement stationnaire et qu'Hercule finit par
  vaincre l'hydre.
\end{abstract}

% Les commandes de sectionnement (\chapter, \section, \subsection, \etc.)
% sont automatiquement numérotés, et permettent de produire facilement
% une table des matières au moyen de :
\tableofcontents

\section{Introduction}
On va parler de suites de Goodstein et d'hydre de Lerne. En particulier on
exposera l'intéressant théorème~\ref{theo-ordinaux} % la commande \ref répond
                                                    % à la commande \label
                                                    % (voir plus bas)
qui contient l'équation~\ref{eq-zero}.  Mais à dire vrai c'est surtout un
prétexte pour donner quelques exemples d'utilisation de \LaTeX. Par exemple
comment fait on pour créer un nouveau paragraphe~?

C'est très simple, on saute une ligne. Cela termine le paragraphe courant et
en commence un nouveau. En \LaTeX{} un blanc est égal à    plusieurs   blancs~;
les blancs en début de ligne sont totalement ignorés, de même que les blancs
suivant un nom de macro% ici on insère une note en bas de page
\footnote{C'est pour cette raison que dans le source \LaTeX{} de ce
  fichier les utilisations de la macro \texttt{$\backslash$LaTeX},
  ainsi que d'autres macros sans arguments sont (presque) toujours
  suivies d'un groupe vide \texttt{\{\}}.%
}~;
un saut de ligne est équivalent à un blanc à la règle ci-dessus près~: deux (ou
plus) sauts de lignes consécutifs ouvrent un nouveau paragraphe.

Enfin tous les caractères suivant un \% et sur la même ligne, 
\emph{y compris le caractère de fin de ligne} s% il n'y aura pas de blanc entre
                                               % ce 's' et le 'o' suivant car
                                               % toutes les fins de lignes sont
                                               % commentées et les blancs en
                                               % début de ligne sont
                                               % ignorés.
                                               % Voici le 'o' maintenant
o% 
n%
t %
ignorés.

\subsection{Caractères accentués}
Une question intéressante~: comment taper des caractères accentués si on n'a
pas un clavier français~? On peut configurer un clavier américain pour obtenir
les caractères accentués mais ça n'est pas toujours simple.

Une autre solution en \LaTeX{} est d'obtenir les accents au moyen de
commandes~: la commande \verb|\'| produit un accent aigu sur la lettre qui
suit, si on tape par exemple \verb|\'elite|, on obtient <<~\'elite~>>. De même
les commandes \verb|\`| et \verb|\^| produisent respectivement un accent grave
et un accent circonflexe (comme dans p\^eche et m\`eche) ; pour obtenir un c
cédille on tape \verb|\c{c}| (fa\c{c}on). Remarquons que ces commandes
fonctionnent quelque soit la lettre que l'on accentue, par exemple on peut
facilement faire \`A ou \~n, voire \'P (P accent aigu) ou \c{O} (O cédille).

Jeu~: deviner quelle commande produit le tréma.

\section{Définitions}
\subsection{Ensemble bien ordonné}

% Utilisation de l'environnement défini dans le préambule

\begin{de}
Un ensemble $X$ muni d'une relation d'ordre $<$ est \emph{bien ordonné} si~:
\begin{itemize}
\item la relation $<$ est totale~;
\item toute partie non vide de $X$ a un plus petit élément.
\end{itemize}
\end{de}

\subsection{Ordinaux}
\begin{de}\label{def-ordinaux} % La commande \label peut se mettre dans
                               % tout contexte numéroté par LaTeX (ici
                               % l'environnement définition)
Un \emph{ordinal} est un ensemble bien ordonné par la relation $\in$.
\end{de}

\subsection{Arithmétique ordinale}
\begin{theo}\label{theo-ordinaux}
Les opérations ordinales vérifient les propriétés suivantes~:
\begin{eqnarray}
  \alpha + 0                     &=& \alpha\label{eq-zero}\\
  (\alpha + \beta)\ordmult\gamma &=& \alpha\ordmult\gamma +
                                     \beta\ordmult\gamma\\
  \alpha^{\beta + 1}             &=& \alpha^\beta\ordmult\alpha
\end{eqnarray}
\end{theo}

\subsection{Exemple de suite de Goostein}
\begin{displaymath}
  \begin{array}{|c|c|c|} % trois colonnes centrées débutant, séparées et
                         % finissant par des lignes verticales
    \hline  % une ligne horizontale pour commencer le tableau
    j & G_i & i\\ % une ligne ; les colonnes sont séparées par &, la ligne est
                  %terminée par \\
    \hline
    \multicolumn{3}{c}{\strut}\\ % une ligne d'une seule colonne (centrée)
                                 % à la place de trois. La commande \strut
                                 % insère une espace vertical correspondant à
                                 % une ligne
    \hline
    2 & 100 & 2\\
    \hline
    3 & 100 & 3\\
    \hline
  \end{array}
  \qquad % insère une espace de deux quadratins entre les deux tableaux
  \begin{array}{|c|c|c|}
    \hline
    j & G_i & i\\
    \hline
    \multicolumn{3}{c}{\strut}\\
    \hline
    2 & 100 & 2\\
    \hline
    3 & 100 & 3\\
    \hline
  \end{array}
\end{displaymath}

\section{Quelques théorèmes qui n'ont rien à voir}

\begin{theo}
  La formule d'Euler~:
  \begin{equation}
    \expo e{2i\pi}=1 % on utilise la macro définie dans le préambule
  \end{equation}
\end{theo}

\begin{prop}
  La somme des $n$ premiers entiers est~:
  \begin{displaymath}
    \sum_{i=1}^ni = \frac{n(n+1)}2
  \end{displaymath}
\end{prop}
% \ref et \pageref permettent de référencer une définition, un théorème ou une
% équation nommé par \label.
Cette propriété est fausse dans le cas des ordinaux (voir
définition~\ref{def-ordinaux}, page~\pageref{def-ordinaux}).

\begin{theo}[Nombre d'or]
  Le développement en fraction continue du nombre d'or est~:
  \begin{equation*} % L'environnement equation numérote
                    % l'équation. L'étoile spécifie de ne pas mettre
                    % de numéro
    \varphi = \cfrac 1{1+\cfrac 1{1 + \cfrac 1{1 + \dotsb}}}
  \end{equation*}
\end{theo}

\begin{theo}\label{theo-log2}
  La fonction $\log_2$ définie par
  \begin{displaymath}
    \log_2 x = \frac{\log x}{\log 2}
  \end{displaymath}
  est l'inverse (à droite) de la fonction <<~2 puissance~>>~/
  \begin{displaymath}
    2^x = e^{x\log 2}
  \end{displaymath}
\end{theo}

En effet on a le calcul suivant~:
% eqnarray est un environnement utile pour aligner des équations. On peut aussi
% utiliser gather pour mettre des formules les unes sous les autres. Comme
% avant l'étoile signifie que l'on ne veut pas numéroter les équations.
\begin{eqnarray*}
 2^{\log_2 x} &=& e^{\log 2\frac{\log x}{\log 2}}\\
              &=& e^{\log x}\\
              &=& x
\end{eqnarray*}

Et voici un dernier petit calcul pour la route, afin de montrer l'usage de
l'environnement \texttt{align}. % \texttt{texte} formatte le texte en police
                                % 'typewriter', c'est à dire à chasse fixe
Ce calcul démontre que dans un anneau commutatif, $0$ (l'élément neutre de
l'addition) est absorbant pour la multiplication~:
% align est un environnement plus souple que eqnarray pour aligner des
% équations ; en particulier le nombre de colonnes n'est pas limité.
\begin{align*}
  0x  &= 0x + 0            & \text{car $0$ est élément neutre de $+$}\\
      &= 0x + (0x + (-0x)) & \text{car $-0x$ est l'opposé de $0x$ pour $+$}\\
      &= (0x + 0x) + (-0x) & \text{par associativité de $+$}\\
      &= (0 + 0)x + (-0x)  & \text{par distributivité}\\
      &= 0x + (-0x)        & \text{car $0$ est neutre, donc $0+0=0$}\\
      &= 0                 & \text{car $-0x$ est l'opposé de $0x$}
\end{align*}
\end{document}

C'est fini ! Tout ce qui vient après \end{document} est ignoré par LaTeX