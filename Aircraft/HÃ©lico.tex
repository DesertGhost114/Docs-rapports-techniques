%\documentclass[12pt,a4paper]{article}

\usepackage[utf8]{inputenc}
\usepackage{lmodern}
\usepackage[T1]{fontenc}
% paysage
% \usepackage[landscape]{geometry}
\usepackage{lscape}
\usepackage{graphicx}
% \graphicspath{ {images/} }

% headers footers
\usepackage{fancyhdr}
\pagestyle{fancy}

% référencer la dernière page
\usepackage{lastpage}

% pdf
\usepackage{pdfpages}

% francais
\usepackage[frenchb]{babel}
% math
\usepackage{amssymb}

\usepackage{multicol}
\usepackage{url}

\usepackage{multido}
\usepackage[utf8]{inputenc}
% \usepackage{lmodern}
\usepackage[T1]{fontenc}

\usepackage[sfdefault]{AlegreyaSans} %% Option 'black' gives heavier bold face
%% The 'sfdefault' option to make the base font sans serif
% \renewcommand*\oldstylenums[1]{{\AlegreyaSansOsF #1}}


\usepackage{multicol}
\usepackage[frenchb]{babel}
% \usepackage{pstricks,pst-plot,pst-node}
% \usepackage{pstricks-add}
\usepackage{pst-circ}
\usepackage{pst-magneticfield}
\usepackage{pst-electricfield}
\usepackage{graphicx}
\usepackage{amsmath,amsfonts,amssymb}
\usepackage{titlesec} 
\usepackage{float}
\usepackage{textcomp}
\usepackage{amssymb}
\usepackage[toc,page]{appendix}
\usepackage{listings} 

\lstset{language=Matlab}
\usepackage{lipsum}
\usepackage{enumerate}


%Numerotation par section des équations
\usepackage{amsmath}

\usepackage{tabularx}
\usepackage{longtable}

%------------------------------inclue les références
% \usepackage[nottoc, notlof, notlot]{tocbibind}
%\usepackage{biblatex}
% \usepackage{csquotes}

%\usepackage{etoolbox}
% \patchcmd{\chapter}{\thispagestyle{plain}}{\thispagestyle{fancy}}{}{}
\title{
	\Huge\textsc{Mécanique de vol d'un hélicoptère}
}
\author{Mohamed Thebti} 

\begin{document}
% retrait de la première ligne d'un paragraphe
\setlength{\parindent}{0mm}

\fancyhead[R]{\slshape \leftmark}
\fancyhead[L]{\slshape Mécanique de vol d'un hélicoptère}
%\fancyhead[LE,RO]{\slshape \rightmark}
% \fancyhead[LO,RE]{\slshape \leftmark}

% \fancyfoot[C]{Travail de Master}
\fancyfoot[L]{\slshape Thebti Mohamed}
\fancyfoot[C]{}
\fancyfoot[R]{\thepage}

\maketitle
\newpage

\tableofcontents

\newpage



\section{Introduction}

But : étudier la mécanique de vol d'un hélicoptère. comme la portance, la stabilité de vol, puissance de vol, ....
\medbreak
Avoir les outils avant d'engager une étude dynamique plus complexe 

\newpage
\section{Description d'un hélicoptère}
pales principales, pales secondaires
angle d'attaque des pales principales

\newpage
\section{Schéma cinématique}


\subsection{Vecteurs positions}
origine : centre de rotation verticale se trouvant sous les pâles principales.
\medbreak
position  des pâles principales ($pp$) : vecteur verticale
\medbreak
position de l'hélice arrière : 
vecteur allant de l'origine vers l'hélice ($h$) arrière. 


\newpage
\section{Angular momentum}

\subsection{Formula}
\begin{equation}
\vec{L}=\vec{OA} \otimes \vec{P}=\vec{r} \otimes \vec{P}=\vec{r} \otimes m \cdot \vec{v}=\vec{I} \otimes \vec{\omega}
\end{equation}
$\vec{L}$ : Angular Momentum [$kg \cdot \frac{m^2}{s}$]\\
$\vec{OA}$ and $r$: position of the mass [$m$] according to a referance\\
$\vec{P}$ : linear momentum [$kg\cdot \frac{m}{s}$]\footnote{$\vec{L}$ is perpendicular to both $\vec{P}$ and $\vec{r}$}\\
$\vec{v}$ : velocity [$\frac{m}{s}$]
$I$ : moment of inertia [$m^2 \cdot kg \cdot$]\\
$\omega$ : angular speed [$\frac{rad}{s}$]

Torque : 
\begin{equation}
M = \frac{d\vec{L}}{dt}=\frac{d(\vec{I} \otimes \vec{\omega})}{dt}
\end{equation}
\medbreak
if we consider a particule of mass $m$, $\vec{r}$ is the position of the center of mass.
If it is a solid object, $L$ is first computed according to the axis of rotation of the object : 
\begin{equation}
\vec{L}_{ar}=\vec{I}_{ar} \otimes \vec{\omega}_{ar}
\end{equation}
To compute the angular moment according to an other axis of rotation (new referance), we use the Huygens-Steiner theorem (or the Parallel axis theorem) : 
\begin{equation}
\vec{L}_{0}=\vec{I}_{0} \otimes \vec{\omega}_{cm}\\
\end{equation}
\begin{equation}
\vec{I}_{0} = \vec{I}_{ar} + m\cdot d^2
\end{equation}
with $d$ the distance between the axis of rotation of the object and the new referance. 
\subsection{Condition of stability}

Main rotor(s):
\begin{equation}
\vec{L}_{mr}=\vec{r}_{mr} \otimes m_{mr} \cdot \vec{v_{mr}}=\vec{I_{mr}} \otimes \vec{\omega_{mr}}
\end{equation}


Rear rotor : 
\begin{equation}
\vec{L}_{rr}=\vec{r}_{rr} \otimes m_{rr} \cdot \vec{v_{rr}}=\vec{I_rr} \otimes \vec{\omega_{rr}}
\end{equation}

assurer la stabilité lors du vol: les moments cinétiques doivent s'annuler. (poser la formule et résoudre)
\begin{equation}
\vec{L_{mr}}=\vec{L_{rr}}
\end{equation}

or

The generated torque is compensated : 
\begin{equation}
\sum \vec{M_{mr}}=\sum \vec{M_{rr}}
\end{equation}

find a relation between $\omega_{mr}$ and $\omega_{rr}$ -> determine the transmission ratio

\subsection{Pivots à droite et à gauche}
pour tourner à gauche ou doite, on ne doit plus satisfaire la condition de stabilité. le pilote utiliser le pédalier pour accélérer/ralentir l'hélice arrière. ainsi les moments cinétiques ne sont plus égaux.
\medbreak
calculer l'effet de rotation sur l'hélicoptère si l'hélice est accélérée/ralentie de 10,20,30,.. $\%$. mettre un tableau. calculer la vitesse de rotation dans ces cas-là. 


\begin{equation}
\begin{bmatrix}
0 \\
0\\
l_1
\end{bmatrix}_{R_{1}} \enspace
\vec{AB}_{R_{2}}=
\begin{bmatrix}
0 \\
l_2\\
0
\end{bmatrix}_{R_{2}} \enspace
\vec{BC}_{R_{3}}=
\begin{bmatrix}
l_3 \\
0\\
0
\end{bmatrix}_{R_{3}} \enspace
\vec{CD}_{R_{4}}=
\begin{bmatrix}
0 \\
0\\
-l_4
\end{bmatrix}_{R_{4}} \enspace
\end{equation}

\begin{itemize}
	\item
	\item 
	\item 
\end{itemize}


\medbreak

\medbreak

\medbreak

\medbreak




\begin{equation}
\begin{split}
\vec{OE}_R=\vec{OA}_R+\vec{AB}_R+\vec{B B_1}_R+\vec{B_1 C_1}_R+\vec{C_1 C}_R\\+\vec{C C_2}_R+\vec{C_2 D}_R+\vec{D D_1}_R+\vec{D_1 E}_R
\end{split}
\end{equation}

\begin{equation}
\begin{split}
\vec{OF}_R=\vec{OA}_R+\vec{AB}_R+\vec{B B_1}_R+\vec{B_1 C_1}_R+\vec{C_1 C}_R\\+\vec{C C_2}_R+\vec{C_2 D}_R+\vec{D D_1}_R+\vec{D_1 E}_R+\vec{E F}_R
\end{split}
\end{equation}

\begin{equation}
\begin{split}
\vec{OG}_R=\vec{OA}_R+\vec{AB}_R+\vec{B B_1}_R+\vec{B_1 C_1}_R+\vec{C_1 C}_R+\vec{C C_2}_R\\+\vec{C_2 D}_R+\vec{D D_1}_R+\vec{D_1 E}_R+\vec{E F}_R+\vec{F F_3}_R+\vec{F_3 G}_R
\end{split}
\end{equation}

\begin{equation}
\begin{split}
\vec{OH}_R=\vec{OA}_R+\vec{AB}_R+\vec{B B_1}_R+\vec{B_1 C_1}_R+\vec{C_1 C}_R+\vec{C C_2}_R+\vec{C_2 D}_R\\+\vec{D D_1}_R+\vec{D_1 E}_R+\vec{E F}_R+\vec{F F_3}_R+\vec{F_3 G}_R+\vec{G H}_R
\end{split}
\end{equation}

 
\section{Conclusion}


\end{document}