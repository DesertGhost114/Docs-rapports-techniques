\documentclass[12pt,a4paper]{article}

\usepackage[utf8]{inputenc}
\usepackage{lmodern}
\usepackage[T1]{fontenc}
% paysage
% \usepackage[landscape]{geometry}
\usepackage{lscape}
\usepackage{graphicx}
% \graphicspath{ {images/} }

% headers footers
\usepackage{fancyhdr}
\pagestyle{fancy}

% référencer la dernière page
\usepackage{lastpage}

% pdf
\usepackage{pdfpages}

% francais
\usepackage[frenchb]{babel}
% math
\usepackage{amssymb}

\usepackage{multicol}
\usepackage{url}

\usepackage{multido}
\usepackage[utf8]{inputenc}
% \usepackage{lmodern}
\usepackage[T1]{fontenc}

\usepackage[sfdefault]{AlegreyaSans} %% Option 'black' gives heavier bold face
%% The 'sfdefault' option to make the base font sans serif
% \renewcommand*\oldstylenums[1]{{\AlegreyaSansOsF #1}}


\usepackage{multicol}
\usepackage[frenchb]{babel}
% \usepackage{pstricks,pst-plot,pst-node}
% \usepackage{pstricks-add}
\usepackage{pst-circ}
\usepackage{pst-magneticfield}
\usepackage{pst-electricfield}
\usepackage{graphicx}
\usepackage{amsmath,amsfonts,amssymb}
\usepackage{titlesec} 
\usepackage{float}
\usepackage{textcomp}
\usepackage{amssymb}
\usepackage[toc,page]{appendix}
\usepackage{listings} 

\lstset{language=Matlab}
\usepackage{lipsum}
\usepackage{enumerate}


%Numerotation par section des équations
\usepackage{amsmath}

\usepackage{tabularx}
\usepackage{longtable}

%------------------------------inclue les références
% \usepackage[nottoc, notlof, notlot]{tocbibind}
%\usepackage{biblatex}
% \usepackage{csquotes}

%\usepackage{etoolbox}
% \patchcmd{\chapter}{\thispagestyle{plain}}{\thispagestyle{fancy}}{}{}
\title{
	\Huge\textsc{Mise en orbite d'une fusée}
}
\author{Mohamed Bechir Thebti} 

\begin{document}
% retrait de la première ligne d'un paragraphe
\setlength{\parindent}{0mm}

\fancyhead[R]{\slshape \leftmark}
\fancyhead[L]{\slshape Etude cinématique d'une pelle mécanique}
%\fancyhead[LE,RO]{\slshape \rightmark}
% \fancyhead[LO,RE]{\slshape \leftmark}

% \fancyfoot[C]{Travail de Master}
\fancyfoot[L]{\slshape Thebti Mohamed}
\fancyfoot[C]{}
\fancyfoot[R]{\thepage}

\maketitle
\newpage

\tableofcontents

\newpage



\section{Introduction}

Etudier la cinématique du lancement et de la mise en orbite d'une fusée
\medbreak
la mise en orbite sera faite en prenant en compte les aspects dynamiques


\newpage
\section{Schémas}
schéma représentatif de la situation

schéma de l'atmosphère, avec formules de températures et pression selon altitude

\section{Etude cinématique}


\subsection{les constantes}

\begin{itemize}
	\item G : 
	\item altitude de la position de départ
	\item k : coefficient caractéristique de la géométrie du solide
	\item $C_x$
	\item S : surface à l'air
	\item 
\end{itemize}


\begin{equation}
\vec{B_1 B_2}_{R_{1}}=
\begin{bmatrix}
v_1\cdot cos(\beta_4) \\
0\\
v_1\cdot sin(\beta_4)
\end{bmatrix}_{R1} \enspace
\vec{B_1 C_1}_{R_{1}}=
\begin{bmatrix}
c_1\cdot cos(\beta_1) \\
0\\
c_1\cdot sin(\beta_1)
\end{bmatrix}_{R1} \enspace
\vec{C_1 C}_{R_{1}}=
\begin{bmatrix}
c\cdot cos(\beta_1) \\
0\\
c\cdot sin(\beta_1)
\end{bmatrix}_{R1} \enspace
\end{equation}

\begin{equation}
\vec{C_1 C_2}=
\begin{bmatrix}
v_2\cdot cos(\gamma_1) \\
0\\
v_2 \cdot sin(\gamma_1)
\end{bmatrix}_{R1} \enspace
\vec{C C_2}_{R_{1}}=
\begin{bmatrix}
-c_2\\
0\\
0
\end{bmatrix}_{R2} \enspace
\vec{C_2 D}_{R_2}=
\begin{bmatrix}
-d\\
0\\
0
\end{bmatrix}_{R2} \enspace
\vec{D D_1}_{R_3}=
\begin{bmatrix}
- d_1\\
0\\
0
\end{bmatrix}_{R_3} \enspace
\end{equation}

\begin{equation}
\vec{D_1 D_2}_{R_3}=
\begin{bmatrix}
-d_2 \cdot cos(\delta_1)\\
0\\
d_2 \cdot sin(\delta_1)
\end{bmatrix}_{R_3} \enspace
\end{equation}

\begin{itemize}
	\item
	\item 
	\item 
\end{itemize}


\medbreak

\medbreak

\medbreak

\medbreak
L : longueur représentative de la fusée


\begin{eqnarray}
ggagsfaaf
\end{eqnarray}

\section{Formules utiles}
nombre Reynolds
\begin{equation}
Re=\frac{\rho_{air} \cdot L \cdot v_{fusee}}{\mu}
\end{equation}
accélération terrestre
\begin{equation}
g= G \frac{m_{terre}}{(rayon_terre+altitude_{z_0}+z)^2}
\end{equation}

Force de pesanteur
\begin{equation}
Fg=m_fusee \cdot g
\end{equation}

frottement laminaire
\begin{equation}
Ffl=-k*\eta*v_{fusee}
\end{equation}

frottement turbulant
\begin{equation}
Fft=-\frac{1}{2} \cdot \rho_{air} \cdot C_x \cdot S \cdot v_{fusee}^2
\end{equation}

quantité de mouvement conservée

\begin{eqnarray}
p_{fusee}=p_{gaz}\\
P_{fusee}=m_{fusee}*v_{fusee}\\
P_{gaz}=m_{gaz}*v_{gaz}\\
dp/dt=somme F=m_{fusee}*a_{fusse}+ d m_{fusee}/dt+v_{fusee}\\
dp/dt=m_gaz*a_gaz+d m_gaz/dt*v_{gaz}=F_{poussee}
\end{eqnarray}
equilibre des forces
\begin{eqnarray}
\sum F=F_{poussée}-F_g-F_{frottement}\\
a=\frac{\sum F}{m_{fusee}}
\end{eqnarray}
Les formules standards de la cinématique donnent vitesse et position. 
\begin{eqnarray}
v_{fusee}=a \cdot \Delta t\\
z=a \cdot \Delta t^2+v_{fusee} \cdot \Delta t+z_0
\end{eqnarray}
\section{Procédure de calcul}

itérer à chaque pas de temps

calculer Re
selon valeur Re, le frottement est laminaire ou turbulant. limite environ vers 3000

force gravité
les frottements
équilibre des forces
accélération
pas de temps
vitesse 
positition

\section{Calcul plus poussé}
masse variable
quantité de mouvement et poussée des gaz. perte de masse à cause des gaz

g change selon l'altitude

les propriétés de l'air change, $\rho$, $\mu$, $\eta$, ...

$L$ de la fusée si elle perd des étages

calcul précis du $C_x$ : cf TAero, ou cours mécanique fluide incompressible EPFL
\section{Conclusion}


\end{document}