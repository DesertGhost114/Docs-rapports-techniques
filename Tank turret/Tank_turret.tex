
\documentclass[12pt,a4paper]{article}

\usepackage[utf8]{inputenc}
\usepackage{lmodern}
\usepackage[T1]{fontenc}
% paysage
% \usepackage[landscape]{geometry}
\usepackage{lscape}
\usepackage{graphicx}
% \graphicspath{ {images/} }

% headers footers
\usepackage{fancyhdr}
\pagestyle{fancy}

% référencer la dernière page
\usepackage{lastpage}

% pdf
\usepackage{pdfpages}



% math
\usepackage{amssymb}

\usepackage{multicol}
\usepackage{url}

\usepackage{multido}
\usepackage[utf8]{inputenc}
% \usepackage{lmodern}
\usepackage[T1]{fontenc}

\usepackage[sfdefault]{AlegreyaSans} %% Option 'black' gives heavier bold face
%% The 'sfdefault' option to make the base font sans serif
% \renewcommand*\oldstylenums[1]{{\AlegreyaSansOsF #1}}


\usepackage{multicol}

% \usepackage{pstricks,pst-plot,pst-node}
% \usepackage{pstricks-add}
\usepackage{pst-circ}
\usepackage{pst-magneticfield}
\usepackage{pst-electricfield}
\usepackage{graphicx}
\usepackage{amsmath,amsfonts,amssymb}
\usepackage{titlesec} 
\usepackage{float}
\usepackage{textcomp}
\usepackage{amssymb}
\usepackage[toc,page]{appendix}
\usepackage{listings} 

\lstset{language=Matlab}
\usepackage{lipsum}
\usepackage{enumerate}


%Numerotation par section des équations
\usepackage{amsmath}

\usepackage{tabularx}
\usepackage{longtable}

%------------------------------inclue les références
% \usepackage[nottoc, notlof, notlot]{tocbibind}
%\usepackage{biblatex}
% \usepackage{csquotes}

%\usepackage{etoolbox}
% \patchcmd{\chapter}{\thispagestyle{plain}}{\thispagestyle{fancy}}{}{}
\title{
	\Huge\textsc{Tank turret rotation}
}
\author{Mohamed Thebti} 

\begin{document}
	% retrait de la première ligne d'un paragraphe
	\setlength{\parindent}{0mm}
	
	\fancyhead[R]{\slshape \leftmark}
	\fancyhead[L]{\slshape Tank turret rotation}
	%\fancyhead[LE,RO]{\slshape \rightmark}
	% \fancyhead[LO,RE]{\slshape \leftmark}
	
	% \fancyfoot[C]{Travail de Master}
	\fancyfoot[L]{\slshape Mohamed Thebti}
	\fancyfoot[C]{}
	\fancyfoot[R]{\thepage}
	
	\maketitle
	\newpage
	
	\tableofcontents
	
	\newpage
	
	
	
	\section{Introduction}
	
	The objective of this report is to evaluate how to control the turret rotation and the gun elevation. 
	
	% \includegraphics[scale=.2]{oxford-2021-01.jpg}
	
	\section{Coordinates}
	Let's consider the trigonometric circle as reference.
	The rotation is positive if it is following the trigonometric direction, which is anti-clockwise.
	
	the turret is at a specified angle, $\alpha_{start}$.
	a target is situated at an angle $\alpha_{finish}$
	the turret must rotate in order to satisfy the following first condition : 
	\begin{equation}
		\alpha_{start} = \alpha_{finish}
	\end{equation}
	The second one is that this movement must take the shortest time. 

	\section{Angle difference}
	The trigonometric circle is divided into 4 quadrants. When calculating the difference between angles, it is useful to know to which quadrant each angle belongs. 
	In case of two neighbor quadrants, the difference is easily found.
	If the angles are in opposite positions (quadrant 1 and 3 or 2 and 4), there is two possibilities to compute the difference. Because we want to shortest path for the turret, the difference of angle should be : 
	\begin{equation}
		\delta = \alpha_{finish} - \alpha_{start} <= \pi
	\end{equation}
	In case the result does not fulfill this condition, $\delta$ must be computed  
	
	
	\newpage	
	
	table
	
	\begin{table}[ht]
		\caption{Holes and threaded holes} % title of Table
		\centering % used for centering table
		\begin{tabular}{c c c} % centered columns (4 columns)
			\hline\hline %inserts double horizontal lines
			Screw & Threaded hole diameter & Bore Hole (clearance) \\ [0.5ex] % inserts table
			%heading
			\hline % inserts single horizontal line
			M3 & 2.8 & 3.2 \\ % inserting body of the table
			4 & 35 & 144 \\
			5 & 45 & 300 \\ [1ex] % [1ex] adds vertical space
			\hline %inserts single line
		\end{tabular}\label{table:nonlin} % is used to refer this table in the text
	\end{table}
	
	Clearance
	To fix two part together, a clearance is needed.
	When two parts must be assembled together, a clearance of 0.1mm is enough. Then use the glue to fix the assembly. 
	
	
	
	\begin{table}[ht]
		\caption{Nonlinear Model Results} % title of Table
		\centering % used for centering table
		\begin{tabular}{c c c c} % centered columns (4 columns)
			\hline\hline %inserts double horizontal lines
			Case & Method\#1 & Method\#2 & Method\#3 \\ [0.5ex] % inserts table
			%heading
			\hline % inserts single horizontal line
			1 & 50 & 837 & 970 \\ % inserting body of the table
			2 & 47 & 877 & 230 \\
			3 & 31 & 25 & 415 \\
			4 & 35 & 144 & 2356 \\
			5 & 45 & 300 & 556 \\ [1ex] % [1ex] adds vertical space
			\hline %inserts single line
		\end{tabular}\label{table:nonlin} % is used to refer this table in the text
	\end{table}
	
	
	
	\begin{equation}
		Length_{lattitude}(\phi) = 111132.92-559.82 \cdot cos(2 \cdot \phi)+1.175*cos(4 \cdot \phi)-0.0023 \cdot cos(6 \cdot \phi)= ... [m/degree]
	\end{equation}
	1 degree longitude at latitude phi
	\begin{equation}
		Length_{longitude}(\phi) =
		111412.84-93.5 \cdot cos(3 \cdot \phi)+ 0.118 \cdot cos(5 \cdot \phi)= ... [m/degree]
	\end{equation}
	
	\newpage
	\section{Schéma cinématique}
	
	
	\subsection{Vecteurs positions}
	origine : centre de rotation verticale se trouvant sous les pâles principales.
	\medbreak
	position  des pâles principales ($pp$) : vecteur verticale
	\medbreak
	position de l'hélice arrière : 
	vecteur allant de l'origine vers l'hélice ($h$) arrière. 
	
	
	\newpage
	\section{Angular momentum}
	
	\subsection{Formula}
	\begin{equation}
		\vec{L}=\vec{OA} \otimes \vec{P}=\vec{r} \otimes \vec{P}=\vec{r} \otimes m \cdot \vec{v}=\vec{I} \otimes \vec{\omega}
	\end{equation}
	$\vec{L}$ : Angular Momentum [$kg \cdot \frac{m^2}{s}$]\\
	$\vec{OA}$ and $r$: position of the mass [$m$] according to a referance\\
	$\vec{P}$ : linear momentum [$kg\cdot \frac{m}{s}$]\footnote{$\vec{L}$ is perpendicular to both $\vec{P}$ and $\vec{r}$}\\
	$\vec{v}$ : velocity [$\frac{m}{s}$]
	$I$ : moment of inertia [$m^2 \cdot kg \cdot$]\\
	$\omega$ : angular speed [$\frac{rad}{s}$]
	
	Torque : 
	\begin{equation}
		M = \frac{d\vec{L}}{dt}=\frac{d(\vec{I} \otimes \vec{\omega})}{dt}
	\end{equation}
	\medbreak
	if we consider a particule of mass $m$, $\vec{r}$ is the position of the center of mass.
	If it is a solid object, $L$ is first computed according to the axis of rotation of the object : 
	\begin{equation}
		\vec{L}_{ar}=\vec{I}_{ar} \otimes \vec{\omega}_{ar}
	\end{equation}
	To compute the angular moment according to an other axis of rotation (new referance), we use the Huygens-Steiner theorem (or the Parallel axis theorem) : 
	\begin{equation}
		\vec{L}_{0}=\vec{I}_{0} \otimes \vec{\omega}_{cm}\\
	\end{equation}
	\begin{equation}
		\vec{I}_{0} = \vec{I}_{ar} + m\cdot d^2
	\end{equation}
	with $d$ the distance between the axis of rotation of the object and the new referance. 
	\subsection{Condition of stability}
	
	Main rotor(s):
	\begin{equation}
		\vec{L}_{mr}=\vec{r}_{mr} \otimes m_{mr} \cdot \vec{v_{mr}}=\vec{I_{mr}} \otimes \vec{\omega_{mr}}
	\end{equation}
	
	
	Rear rotor : 
	\begin{equation}
		\vec{L}_{rr}=\vec{r}_{rr} \otimes m_{rr} \cdot \vec{v_{rr}}=\vec{I_rr} \otimes \vec{\omega_{rr}}
	\end{equation}
	
	assurer la stabilité lors du vol: les moments cinétiques doivent s'annuler. (poser la formule et résoudre)
	\begin{equation}
		\vec{L_{mr}}=\vec{L_{rr}}
	\end{equation}
	
	or
	
	The generated torque is compensated : 
	\begin{equation}
		\sum \vec{M_{mr}}=\sum \vec{M_{rr}}
	\end{equation}
	
	find a relation between $\omega_{mr}$ and $\omega_{rr}$ -> determine the transmission ratio
	
	\subsection{Pivots à droite et à gauche}
	pour tourner à gauche ou doite, on ne doit plus satisfaire la condition de stabilité. le pilote utiliser le pédalier pour accélérer/ralentir l'hélice arrière. ainsi les moments cinétiques ne sont plus égaux.
	\medbreak
	calculer l'effet de rotation sur l'hélicoptère si l'hélice est accélérée/ralentie de 10,20,30,.. $\%$. mettre un tableau. calculer la vitesse de rotation dans ces cas-là. 
	
	
	\begin{equation}
		\begin{bmatrix}
			0 \\
			0\\
			l_1
		\end{bmatrix}_{R_{1}} \enspace
		\vec{AB}_{R_{2}}=
		\begin{bmatrix}
			0 \\
			l_2\\
			0
		\end{bmatrix}_{R_{2}} \enspace
		\vec{BC}_{R_{3}}=
		\begin{bmatrix}
			l_3 \\
			0\\
			0
		\end{bmatrix}_{R_{3}} \enspace
		\vec{CD}_{R_{4}}=
		\begin{bmatrix}
			0 \\
			0\\
			-l_4
		\end{bmatrix}_{R_{4}} \enspace
	\end{equation}
	
	\begin{itemize}
		\item
		\item 
		\item 
	\end{itemize}
	
	
	\medbreak
	
	\medbreak
	
	\medbreak
	
	\medbreak
	
	
	
	
	\begin{equation}
		\begin{split}
			\vec{OE}_R=\vec{OA}_R+\vec{AB}_R+\vec{B B_1}_R+\vec{B_1 C_1}_R+\vec{C_1 C}_R\\+\vec{C C_2}_R+\vec{C_2 D}_R+\vec{D D_1}_R+\vec{D_1 E}_R
		\end{split}
	\end{equation}
	
	\begin{equation}
		\begin{split}
			\vec{OF}_R=\vec{OA}_R+\vec{AB}_R+\vec{B B_1}_R+\vec{B_1 C_1}_R+\vec{C_1 C}_R\\+\vec{C C_2}_R+\vec{C_2 D}_R+\vec{D D_1}_R+\vec{D_1 E}_R+\vec{E F}_R
		\end{split}
	\end{equation}
	
	\begin{equation}
		\begin{split}
			\vec{OG}_R=\vec{OA}_R+\vec{AB}_R+\vec{B B_1}_R+\vec{B_1 C_1}_R+\vec{C_1 C}_R+\vec{C C_2}_R\\+\vec{C_2 D}_R+\vec{D D_1}_R+\vec{D_1 E}_R+\vec{E F}_R+\vec{F F_3}_R+\vec{F_3 G}_R
		\end{split}
	\end{equation}
	
	\begin{equation}
		\begin{split}
			\vec{OH}_R=\vec{OA}_R+\vec{AB}_R+\vec{B B_1}_R+\vec{B_1 C_1}_R+\vec{C_1 C}_R+\vec{C C_2}_R+\vec{C_2 D}_R\\+\vec{D D_1}_R+\vec{D_1 E}_R+\vec{E F}_R+\vec{F F_3}_R+\vec{F_3 G}_R+\vec{G H}_R
		\end{split}
	\end{equation}
	
	
	\section{Conclusion}
	
	
\end{document}