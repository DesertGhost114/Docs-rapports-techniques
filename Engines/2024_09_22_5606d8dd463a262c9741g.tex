\documentclass[10pt]{article}
\usepackage[utf8]{inputenc}
\usepackage[T1]{fontenc}
\usepackage{amsmath}
\usepackage{amsfonts}
\usepackage{amssymb}
\usepackage[version=4]{mhchem}
\usepackage{stmaryrd}

\begin{document}
\section*{1 Introduction}
The main objective of this report is compute the thermodynamics cycle of a combustion engine and the corresponding output power generated. 
Secondary objective : torque profile $T(RPM)$, efficiency $\eta$, fuel consumption profile according to the engine RPM.

\section*{2 Engine power}
\begin{enumerate}
  \item compute pressure and power with thermodynamic cycle
\end{enumerate}

Study the combustion to estimate its efficiency:
\begin{equation*}
\eta_{\text {combustion }}=\frac{q_{\text {out }}}{q_{\text {in }}} \tag{1}
\end{equation*}


$q$ : energy $[J / kg]$

The results is per $[\mathrm{kg}] \rightarrow \mathrm{J} / \mathrm{kg}, W / \mathrm{kg}, \ldots$\\
2. choose the mass flow $\dot{m}$, the displacement, and size of combustion chamber. -> we can compute power in $W$ and pressure in Pa\\
$V_{\min }$ is the volume when the piston is at TDC\\
$V_{\text {max }}$ is the volume when the piston is at BDC\\
3. Establish the thrust force, along the $y$-axis $R_{1}$ : main Referential, centered on the crankshaft :\\

\begin{itemize}
	\item $x$ to the right
  \item y to the top
  \item z perpendicular to the plan formed by x and y .
  \item The piston head is moving in the $y$ direction.
\end{itemize}


\begin{gather*}
F_{\text {thrust }}=p \cdot S_{p}=p \cdot \pi \cdot\left(\frac{D}{2}\right)^{2}  \tag{2}\\
\vec{F}_{\text {piston }}=\left[\begin{array}{c}
0 \\
-1 \\
0
\end{array}\right]_{R_{1}} \tag{3}
\end{gather*}


$F_{\text {thrust }}$ : thrust force $[N]$\\
$p$ : pressure inside the combustion chamber $[P a]$\\
$D:$ piston diameter, bore $[m]$\\
$S_{p}:$ piston head section $\left[m^{2}\right]$\\
4. subtract the power lost because of friction between piston head and cylinder:


\begin{equation*}
F_{\text {thrust }_{2}}=F_{\text {thrust }}-F_{\text {frictionpiston-cyclinder }} \tag{4}
\end{equation*}


\begin{enumerate}
  \setcounter{enumi}{4}
  \item the thrust fore in the connecting rod referential
\end{enumerate}

\[
\vec{F}_{\text {thrust }}^{R_{c c}}=F_{\text {thrust }_{2}} \cdot\left[\begin{array}{c}
\cos \left(\theta_{c c}\right)  \tag{5}\\
\sin \left(\theta_{c c}\right) \\
0
\end{array}\right]_{R_{c c}}
\]

$\theta_{c c}$ : angle between connecting rod and the $y$ axis\\
6. subtract the power/energy lost because of between the crankshaft and oil


\begin{equation*}
\vec{F}_{\text {thrust }}^{\text {total }}=\vec{F}_{\text {thrust }}^{R_{c c}}+\vec{F}_{\text {oilfriction }} \tag{6}
\end{equation*}


Here, we can also add the energy consumed by the running gears : water pump, oil pump, alternator,\\
7. Torque is the vector product of this force with the position of the connecting rod, from the center of the crankshaft. It should be oriented in the Z-direction.

\[
\vec{T}_{\text {piston }}=\vec{F}_{\text {thrust }}^{\text {total }} \times \text { position }_{c c}=\vec{F}_{\text {thrust }}^{\text {total }} \times\left[\begin{array}{c}
\cos \left(\alpha_{c c}\right)  \tag{7}\\
\sin \left(\alpha_{c c}\right) \\
0
\end{array}\right]_{R_{1}}
\]

$\alpha_{c c}$ : connecting rod angle according to $R_{1}$\\
8. Power output is the scalar product of the torque with crankshaft angular speed


\begin{equation*}
\text { Power }=\vec{T}_{\text {piston }} \cdot \text { ome } \vec{e} g a \tag{8}
\end{equation*}


\begin{enumerate}
  \setcounter{enumi}{8}
  \item Engine mechanical efficiency: quotient of output power by the inputheat power (fuel power).
\end{enumerate}


\begin{equation*}
\eta_{m}=\frac{\text { Power }}{\dot{Q}_{\text {in }}} \tag{9}
\end{equation*}


\begin{enumerate}
  \setcounter{enumi}{9}
  \item link cycle phases with 4 strokes with crankshaft angle.
  \item then follow steps defined in Notion
\end{enumerate}

\subsection*{2.0.1 Friction}
different frictions, which increase the mechanical loses

\begin{itemize}
  \item Piston friction : cylinder, piston pin,
\end{itemize}

\[
\overrightarrow{B C}_{R_{1}}=\left[\begin{array}{c}
\overline{B C}_{x_{1}}  \tag{10}\\
\overline{B C}_{y_{1}} \\
0
\end{array}\right]_{R_{1}}
\]

\subsection*{2.1 Engine thrust}
Theorem of the preservation of linear momentum:


\begin{equation*}
\iint_{\Sigma}[P \cdot \vec{n}+\rho \cdot \vec{V}(\vec{V} \cdot \vec{n})]=0 \tag{11}
\end{equation*}


The term $\vec{V} \cdot \vec{n}$ is equal to zero when the velocity is perpendicular to the normal vector, which is the case on the lateral surface. What remain are the terms corresponding to the surfaces $A_{0}$ and $A_{10}$, which are perpendicular to the axis of the jet engine. The thrust:


\begin{align*}
F & =\left(P_{10}+\rho_{10} \cdot V_{10}^{2}\right) A_{10}-\left(P_{0}+\rho_{0} \cdot V_{0}^{2}\right) A_{0}  \tag{12}\\
& =P_{10} \cdot A_{10}+\rho_{10} \cdot V_{10}^{2} \cdot A_{10}-P_{0} \cdot A_{0}-\rho_{0} \cdot V_{0}^{2} \cdot A_{0}
\end{align*}


mass flow:


\begin{gather*}
\dot{m}=D=\rho \cdot V \cdot A  \tag{13}\\
F=P_{10} \cdot A_{10}+D_{10} \cdot V_{10}-P_{0} \cdot A_{0}-D_{0} \cdot V_{0}  \tag{14}\\
=D_{10} \cdot V_{10}-D_{0} \cdot V_{0}+P_{10} \cdot A_{10}-P_{0} \cdot A_{0}
\end{gather*}


By using the simple trick of adding $P_{0} \cdot A_{10}-P_{0} \cdot A_{10}=0$


\begin{align*}
F & =D_{10} \cdot V_{10}-D_{0} \cdot V_{0}+P_{10} \cdot A_{10}-P_{0} \cdot A_{0}+P_{0} \cdot A_{10}-P_{0} \cdot A_{10}  \tag{15}\\
& =D_{10} \cdot V_{10}-D_{0} \cdot V_{0}+\left(P_{10}-P_{0}\right) A_{10}+P_{0}\left(A_{10}-A_{0}\right)
\end{align*}


It is useful to add the drag due to the engine housing:


\begin{equation*}
X_{c}=-\iint P_{c} \cdot d A \vec{n} \vec{x}->X_{c}=P_{c}\left(A_{10}-A_{0}\right) \tag{16}
\end{equation*}


Where the $P_{c}$ is the pressure at the surface of the housing. Historically, the first jet fighter, the Me-262, was equipped with two Junkers Jumo 004 B turbine engines with housings. So, these housings will produce each a drag force. Since then, engines were housed inside the body of the jet fighter. In this case, this term is equal to 0 .


\end{document}