%%%%%%%%%%%%%%%%%%%%%%%%%%%%%%%%%%%%%%%%%
% Minimalist Book Title Page 
% LaTeX Template
% Version 1.0 (27/12/12)
%
% This template has been downloaded from:
% http://www.LaTeXTemplates.com
%
% Original author:
% Peter Wilson (herries.press@earthlink.net)
%
% License:
% CC BY-NC-SA 3.0 (http://creativecommons.org/licenses/by-nc-sa/3.0/)
% 
% Instructions for using this template:
% This title page compiles as is. If you wish to include this title page in 
% another document, you will need to copy everything before 
% \begin{document} into the preamble of your document. The title page is
% then included using \titleTH within your document.
%
%%%%%%%%%%%%%%%%%%%%%%%%%%%%%%%%%%%%%%%%%

%----------------------------------------------------------------------------------------
%	PACKAGES AND OTHER DOCUMENT CONFIGURATIONS
%----------------------------------------------------------------------------------------

% packages
\documentclass[12pt,a4paper,twoside]{article}

\usepackage[svgnames]{xcolor} % Required to specify font color

\newcommand*{\plogo}{\fbox{$\mathcal{PL}$}} % Generic publisher logo


\usepackage[utf8]{inputenc}
%\usepackage[latin1]{inputenc} 
\usepackage[T1]{fontenc}
\usepackage{lmodern} % load a font with all the characters
\usepackage{gensymb}
\usepackage{mathtools,xparse}

\usepackage{amsthm}

\usepackage{amsmath}

\usepackage{amssymb}
\usepackage{pdfpages}
\usepackage{mathrsfs}

\usepackage{graphicx}

\usepackage{hyperref}

% inclure the numéro du chapitre dans les équations
\numberwithin{equation}{subsection}

\DeclarePairedDelimiter{\abs}{\lvert}{\rvert}
\DeclarePairedDelimiter{\norm}{\lVert}{\rVert}


%----------------------------------------------------------------------------------------
%	TITLE PAGE
%----------------------------------------------------------------------------------------

\newcommand*{\titleTH}{\begingroup % Create the command for including the title page in the document
	\raggedleft % Right-align all text
	\vspace*{\baselineskip} % Whitespace at the top of the page
	
	{\Large Mohamed Thebti}\\[0.167\textheight] % Author name
	
	{\LARGE\bfseries Résumé du cours}\\[\baselineskip] % First part of the title, if it is unimportant consider making the font size smaller to accentuate the main title
	
	{\textcolor{Red}{\Huge Mécanique des fluides}}\\[\baselineskip] % Main title which draws the focus of the reader
	
	{\Large \textit{de Prof. Claude-André Porret}}\par % Tagline or further description
	
	\vfill % Whitespace between the title block and the publisher
	
	{\large Haute école de l'Arc jurassien (HE-Arc)}\par % Publisher and logo
	
	\vspace*{3\baselineskip} % Whitespace at the bottom of the page
	\endgroup}

%----------------------------------------------------------------------------------------
%	BLANK DOCUMENT
%----------------------------------------------------------------------------------------

\begin{document} 
	\pagestyle{empty} % Removes page numbers
	
	\titleTH % This command includes the title page
	
	
	% indentation des paragraphes
	\setlength{\parindent}{0cm}
	
	\newpage
	\tableofcontents
	
	\newpage
	\section{Équations de conservation}
	débit massique
	\begin{equation}
	\dot{m}=\frac{dm}{dt}=\rho \cdot S \cdot v
	\end{equation}
	cas stationnaire : on a la conservation de la masse/débit
	\begin{eqnarray}
	m_{in}=m_{out}\\
	\rho_1 \cdot S_1 \cdot v_1=\rho_2 \cdot S_2 \cdot v_2
	\end{eqnarray}

	Dans le cas d'un fluide incompressible : 
	\begin{equation}
	\rho_1=\rho_2=\rho = cst
	\end{equation}
	Conservation du débit (sans pertes) : 
	
	\begin{equation}
	S_1 \cdot v_1=(S_2 \cdot v_2+S_3 \cdot v_3)
	\end{equation}
	
	Conservation de l'énergie
	Énergie cinétique + Énergie potentielle + Énergie de pression = densité de pression
	
	Calcul pression vérin hydraulique/pneumatique
	
	pompe hydraulique
	
	\end{document}